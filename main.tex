\documentclass[twoside,11pt]{article}

% Any additional packages needed should be included after jmlr2e.
% Note that jmlr2e.sty includes epsfig, amssymb, natbib and graphicx,
% and defines many common macros, such as 'proof' and 'example'.
%
% It also sets the bibliographystyle to plainnat; for more information on
% natbib citation styles, see the natbib documentation, a copy of which
% is archived at http://www.jmlr.org/format/natbib.pdf

\usepackage{jmlr2e}

% Definitions of handy macros can go here

\newcommand{\dataset}{{\cal D}}
\newcommand{\fracpartial}[2]{\frac{\partial #1}{\partial  #2}}
\newcommand{\code}[1]{\texttt{#1}}

% Heading arguments are {volume}{year}{pages}{date submitted}{date published}{paper id}{author-full-names}

% TODO: update heading
\jmlrheading{1}{2000}{1-48}{4/00}{10/00}{meila00a}{Marina Meil\u{a} and Michael I. Jordan}

% Short headings should be running head and authors last names
% TODO: update short heading
\ShortHeadings{TorsionNet}{Jiang, ...}
\firstpageno{1}

\begin{document}

\title{TorsionNet}

% TODO: update authors
\author{\name Runxuan Jiang \email runxuanj@umich.edu \\
       \name Tarun Gogineni \email tgog@umich.edu \\
       \name Josh \\
       \name Ambuj \\
       \name Paul \\
       \addr
       University of Michigan\\
       Ann Arbor, MI 48109, USA} 
% TODO: update editors
\editor{TBD}

\maketitle

\begin{abstract}%   <- trailing '%' for backward compatibility of .sty file
  We propose \code{TorsionNet}, an open-source deep reinforcement learning library designed for molecular conformer generation using Python. \code{TorsionNet} features several pre-built environments and baseline agents based on state-of-the-art algorithms in the field. The environments and agents are built on a modular interface, allowing users to easily build and test new implementations. Additionally, \code{TorsionNet} comes with extensive logging and visualization tools for evaluation of agents and generated conformers, as well as a toolkit for generating and modifying molecules. \code{TorsionNet} is well-tested and thoroughly documented, and is available through on PyPi and on Github: \code{https://github.com/ZimmermanGroup/conformer-ml}.
\end{abstract}

\begin{keywords}
  reinforcement learning, deep learning, deep reinforcement learning, open source, conformer generation, computational chemistry
\end{keywords}

\section{Introduction}
In the past decade, deep reinforcement learning has been increasingly explored in the field of computational chemistry,
including tasks such as protein folding [], improving chemical reactions [], and conformer generation [].

\section{TorsionNet Architecture}
pass

\subsection{Environments}
pass

\subsection{Molecule Generation}
pass

\subsection{Agents}
pass

\subsection{Models}
pass

\subsection{Logging and Analysis}
pass
\section{Future Work}
pass
\section{Conclusions}
pass



% Acknowledgements should go at the end, before appendices and references

\acks{We would like to acknowledge ...}

\vskip 0.2in
\bibliography{reference}


% Manual newpage inserted to improve layout of sample file - not
% needed in general before appendices/bibliography.

\newpage

\appendix
\section*{Appendix A.}

\end{document}
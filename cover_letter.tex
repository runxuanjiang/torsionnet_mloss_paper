\documentclass[twoside,11pt]{article}

\usepackage{jmlr2e}

\begin{document}

    % TODO: update names of editors
    \noindent Dear Editors: \\
    
    We would like to submit our manuscript "Conformer-RL: A Deep Reinforcement Learning Library for Conformer Generation" as well as the source code for the Conformer-RL library for publication in the machine learning open source software section (MLOSS) section of the Journal of Machine Learning Research. The software library Conformer-RL is an open source Python library for the application of deep reinforcement learning in the chemoinformatic task of conformer generation. The library is released under the open-source MIT license. The library is available publicly on both Github: \url{https://github.com/ZimmermanGroup/conformer-rl} and PyPI: \url{https://pypi.org/project/conformer-rl/}. Documentation and a full API reference is also available online at \url{https://conformer-rl.readthedocs.io/en/latest/}, with additional developer documentation included in the README.md file at the root of the project. The library is well-tested with a code coverage of around 96\%. The version of the code we would like to have reviewed is version 0.1.0 on PyPI or the master branch on Github. 
    
    \medskip

    Many of the methods implemented in the Conformer-RL library were first introduced in the paper "TorsionNet: A Reinforcement Learning Approach to Sequential Conformer Search" in the 34th Conference on Neural Information Processing Systems (NeurIPS 2020). This library provides a modular framework that can be used to reproduce the results found in TorsionNet, as well as building and running new experiments within the conformer generation task.

    \medskip
    
    % Todo: Suggestions for editors.
    Our suggestions for reviewers include:
    \begin{itemize}
        \item Joshua Schrier (\url{jschrier@fordham.edu}) for his previous work in using machine learning approaches in materials discovery.
        \item Bharath Ramsundar (\url{bharath.ramsundar@gmail.com}) who created deepchem.io, a similar deep learning library for drug discovery.
        \item Matthias Fey (\url{matthias.fey@tu-dortmund.de}) and Jan Eric Lenssen (\url{janeric.lenssen@udo.edu}) who developed Pytorch Geometric, a graph neural network library used in this work.
    \end{itemize}

    
    
    \medskip
    
    All authors of the paper consent to its submission and review, and the paper and source code have not been submitted to any other journals or conferences. We have no conflicts of interest to disclose.
    
    \medskip
    
    \noindent Thank you for your consideration,
    
    \medskip

    \noindent Runxuan Jiang \\
    Tarun Gogineni \\
    Joshua Kammeraad \\
    Yifei He \\
    Ambuj Tewari \\
    Paul Zimmerman
    
    

\end{document}